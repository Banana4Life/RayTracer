\section{Diskussion}

Der entwickelte Ansatz deckt sehr viel der gegebenen Problemstellung ab. Schatten werden im zweidimensionalen gezeichent
und weisen weich verlaufende Kanten auf. Verschiedene Objekte und Lichter können in die Szene gesetzt
werden. Für viele Anwendungen zum Beispiel im Videospielebereich reicht dieser Ansatz aus. Jedoch
muss die Lösung spätestens bei der Übersetzung in den dreidimensionalen Raum erweitert werden.

\subsection{Grenzen des gewählten Ansatz}

Größter Kritikpunkt an dem gezeigten Verfahren ist die Beschränkung auf den zweidimensionalen Raum.
Für einige Spiele- und Userinterfaceprojekte würden diese Schatten ausreichen, jedoch wird heutzutage in Filmen und großen Videospieleproduktionen
viel Wert auf möglichst realitätsnahe computergenerierte Grafiken
gelegt. Dafür werden die in Kapitel \ref{sec:methods} beschriebenen Methoden benötigt.

Die entwickelte Anwendung ist auf Rechtecke im Raum beschränkt, die an den Koordinatenachsen
ausgerichtet sind. Jedoch kann diese signifikante Einschränkung behoben werden, da der
momentane Programmcode mit leichten Modifikationen auch andere Formen unterstützen kann.

Die Performance der Berechnungen ist ein weiterer Kritikpunkt der entwickelten Methode. Die Laufzeit
steigt zwar in Abhängigkeit der Lichter als auch der Objekte auf dem Bildschirm linear an, jedoch
ist Berechnungsdauer bei mehreren hundert Lichtern und Objekten im Verhältnis zur Komplexität des
Algorithmus zu hoch um diese Methode mit komplexen Szenen in Echtzeitanwendungen einsetzen zu können.

Der entwickelte Algorithmus ist durch die genannten Einschränkungen am Besten in zweidimensionalen
Videospielen einsetzbar. Dort können solche Berechnungen für das Sichtfeld des Spielers oder wie in
der entwickelten Anwendung zur Schattenberechnung eingesetzt werden. Hier kann es auch interessant
sein die Schattenberechnung zu erweitern und Licht komplexer darzustellen. Der erste Schritt zur
Erweiterung in diese Richtung wäre die Einführung von farbigen Lichtern. Somit könnten farbliche
Überlagerungen zur Erzeugung interessanter Effekte genutzt werden.

Weiterhin könnten die schon implementierten Materialien, welche bis jetzt nur zur Speicherung der
Farbe der Objekte in der Szene genutzt werden, mit Transparenz erweitert werden. Verschiedene
Materialien brechen Licht in unterschiedlichen Weisen.

Die Schattenberechnungsmethode bietet Möglichkeiten Lichtstreuung und -reflektion zu
implementieren. Auch hierzu müssten die Materialien erweitert werden, jedoch könnten dann realistischere und grafisch ansprechendere Szenen dargestellt werden.

Eine Problematik die auch in den heutigen sehr realistischen Schattenberechnungsalgorithmen bleibt ist die Lichtbeugung an Objektkanten. Diesen Effekt nähert der entwickelte Algorithmus mit der Nutzung vieler Punktlichtquellen anstelle einer flächenartigen Lichtquelle an. Darüber hinaus hat der Effekt in den meisten praktischen Anwendungen keine Bedeutung, da er nur sehr schwach und für das menschliche Auge nicht sichtbar ist.

\subsection{Alternative Ansätze}

Bei komplexen Szenen stößt die hier gezeigte Implementierung des Algorithmus durch verschiedene Faktoren an ihre Grenzen. Jedoch kann dieses Problem
gelöst werden. Zum Einen gibt es die Möglichkeit Schatten komplett oder teilweise statisch zu berechnen.
Dies ist möglich, wenn sich die Lichtquellen oder Objekte im Raum nicht oder nur teilweise bewegen. Das
ist zum Beispiel der Fall wenn sich ein Objekt innerhalb des beschatteten Bereich bewegt ohne eine der
Schattenkanten zu berühren.
In jedem Fall können komplett statische Lightmaps, welche zu jedem Pixel in der Szene einen
Beleuchtungswert speichern, mit dynamischen Lightmaps verrechnet werden um Rechenaufwand zu sparen. Weiterhin könnte den Rasterizer der Grafikkarte genutzt werden um Polygone auf den Bildschirm zu zeichnen. Dieser ist trotz des momentan verwendeten sehr effizienten aktiven Kantenlisten Algorithmus viel schneller, da er die Beschleunigung durch die Grafikkarte nutzt. Weitere Optimierungen sind möglich.

\subsubsection*{Ray casting}

Ein alternativer Ansatz wäre "`Ray casting"' gewesen, eine einfache Form des Ray tracing. Im zweidimensionalen
Raum werden dabei eine bestimmte festdefinierte Anzahl an Strahlen in alle Richtungen von jeder Lichtquelle
aus geschossen. Dabei laufen diese Strahlen bis zum ersten Punkt an dem sie mit einem Objekt kollidieren
oder bis zum Rand des Bildes beziehungsweise des Simulationsbereichs.

Im Vergleich zu dem von gewählten Verfahren müssten hier bedeutend mehr Strahlen berechnet werden um die
selben präzisen Schatten zu erhalten. Je nach Distanz, die die Strahlen zurücklegen müssen, muss die Anzahl
der Strahlen beim Ray casting erhöht werden um weiterhin genaue Schatten liefern zu können, denn die Strahlen
laufen gemäß der Kreisumfangsformal mit steigendem Radius auseinander. Auf das implementierte Verfahren hat
die Distanz zur Lichtquelle keinen Einfluss, denn es werden direkt die Ecken der Objekte betrachtet.

\subsection{Abdeckung der Problematik}

Eine Schattenberechnung im zweidimensionalen Raum wurde implementiert. Diese Schatten umfassen je nach Konstellation der Objekte und Lichter Umbra, Penumbra
und Antumbra. Auch harte Schatten werden durch die Annäherung
großer flächenartiger Lichtquellen durch viele kleine Punktlichter auf elegante Art und Weise vermieden und realitätsnah
dargestellt. Für viele Videospiele im zweidimensionalen Raum reicht diese Art von Schattenberechnung aus,
jedoch bleibt Erweiterungsspielraum. Weitere Implementierungen von farbigen Lichtern und die Unterstützung
von komplexeren Materialien für die Objekte im Raum sind für die Zukunft denkbar und mit dem entwickelten Code möglich.
