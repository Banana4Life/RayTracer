\section{Durchführung}

Hier ein paar Regeln zum Schreiben eines Artikels, 
gegen die ich teilweise in dieser Ausarbeitung bereits
verstoßen habe.
\begin{list}{-}{}
\item[(a)] Schreiben Sie nicht in 'Ich'-Form! Versuchen Sie entweder 
unpersönlich zu bleiben oder ggf. im Namen des Forschungsteams mit 'wir' zu arbeiten.

\item[(b)] Sprechen Sie den Leser nicht an! Der Leser soll sich selber ein
objektives Bild über den Artikel machen. Ich hoffe Sie verstehen das, oder?

\item[(c)] Bilder, Diagramme und Tabellen stehen alleine. Im Text findet man
nur die Referenzen darauf, wie Sie dies mit Bezug auf
Abb.~\ref{fig:audio} ersehen.

\item[(d)] Bilder, Diagramme und Tabellen sind nie vor der Seite der entsprechenden 
Referenz im Text zu finden. Meistens findet man sie gesammelt im oberen
Seitenbereich, der entsprechenden oder der nahen Folgeseiten. Die Positionierung
im oberen Bereich der Seite hilft den Schnelldurchblätterer, dass seine 
Augen nicht so viel springen müssen.

\item[(e)] Gleichungen und Formeln wie beispielsweise das Faltungsintegral
\begin{equation}
g(t) = \int_{-\infty}^{+\infty}\!\!d\tau \; h(\tau) \, f(t-\tau)
\label{eq:faltung}
\end{equation}
stehen niemals isoliert und sollen sich harmonisch in die Satzstruktur einfügen.
Es empfiehlt sich bei neuen Variablen gleich unter der Gleichung, diese
zu benennen bzw. zu erklären, damit der Leser nicht lange suchen muss.

\item[(f)] Bezieht man sich auf eine Formel, weil man beispielsweise findet,
dass die Integralschreibweise für die Korrelationsanalyse 
\begin{equation}
g(t) = \int_{-\infty}^{+\infty}\!\!d\tau \; h(\tau) \, f(t + \tau)
\end{equation}
dem Faltungsintegral aus Gl.~\eqref{eq:faltung} sehr ähnlich sieht,
so sollte man anhand der hier getätigten Referenz die Umsetzung ersehen.

\item[(g)] Zitieren Sie jemanden oder beziehen Sie sich nur auf eine
Veröffentlichung wie beispielsweise auf ein interessantes Experiment 
zur Totalreflektion \cite{Goos1947}, so können Sie dies beispielsweise 
durch Nummern tun, die im Literaturverzeichnis ausgeführt sind.
Andernorts findet man auch gern den Nachnamen des Erstautors und
das Erscheinungsjahr als Schlüssel, um den Eintrag im Literaturverzeichnis
zu finden. 

Übrigens findet man in naturwissenschaftlichen Schriften 
äußerst selten wörtliche Zitate und meist Referenzen, oder wollten Sie
aus der Arbeit von Einstein zur Speziellen Relativitätstheorie \cite{Einstein1905}
wörtlich zitieren?

\item[(h)] Das Literaturverzeichnis sollte ausreichend Informationen enthalten,
um die Veröffentlichungen auch zu finden. Ebenso muß es einheitlich für
die einzelnen Veröffentlichungstypen wie Buch, Artikel, Webseite etc. sein,
denn nur die Wenigsten mögen es, sich durch scheinbares Chaos fremder Leute zu wühlen.

\item[(i)] Argumentieren Sie! Entschuldigen Sie sich nicht für Ihre Arbeit, 
aber argumentieren und diskutieren Sie die Dinge, die merkwürdig sind.
Führen Sie weitestgehend objektive Gründe an, wenn etwas nicht funktionierte.
Argumentationen, Begründungen etc. müssen meist nicht lang sein, 
aber über offensichtliche Wiedersprüche zu schweigen wirkt unprofessionell 
und der verärgerte Leser wird vielleicht noch unliebsamere Worte 
zum Lästern finden.

\item[(j)] Vermeiden Sie Füllwörter! Obwohl man aber vielleicht auch behaupten
könnte, dass dann auch ein wenig mehr Text gefüllt wird.
\end{list}

\begin{figure}[t]
\centering
%\includegraphics[width=0.45\textwidth]{Bilder/audio-signal-short.pdf}
\caption{Das akustische Signal von etwas Gestammelten, wobei die Amplituden in 
nicht näher zu bezeichnenden Einheiten (a.u. für arbitrary unit) angegeben sind.}
\label{fig:audio}
\end{figure}
