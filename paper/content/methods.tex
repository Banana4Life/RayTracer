\section{Problemstellung und Methoden}

Das Problem ist, im zweidimensionalen Raum realistische Schatten zu berechnen möglichst die, wie echte
Schatten aus den drei Teilen ``Umbra'', ``Penumbra'' und ``Antumbra''

\subsection{Umbra, Penumbra und Antumbra}

In der Theorie von Schatten wird ein Schatten in 4 Teile zerlegen.
\begin{enumerate}
 \item \emph{Umbra}, der Kernschatten, ist der Teil des Schattens der von keinerlei Licht erreicht wird.
       Wenn es nur eine Lichtquelle gibt, dann hat jeder Schatten einen Umbra.
 \item \emph{Antumbra}, der weiche Schatten, der hinter dem Umbra entsteht. Dieser Bereich entsteht nur,
       wenn die Lichtquelle größer ist, als das Objekt, dass den Schatten wirft. Der Antumbra beginnt an
       dem Punkt, an dem das Objekt die Lichtquelle nicht mehr vollständig verdecken kann und man die
       Ränder der Lichtquelle sehen kann.
 \item \emph{Penumbra}, die zwei weichen Schatten, die an den Seiten des Umbras entstehen. Diesen Bereich
       gibt es immer, wenn der Schatten nicht von einer Punktlichtquelle geworfen wird, in der realen
       Welt als immer. In diesem Bereich beginnt man die Seite der Lichtquelle zu sehen, die vorher im
       Umbra verdeckt wurde.
\end{enumerate}
Diese 3 separaten Bereiche eines Schatten sollten optisch realistisch dargestellt werden können.
Damit gilt also als Anforderung, dass Lichtquellen simuliert werden können, die eine Breite haben
und nicht nur einfach ein Punkt sind.


\subsection{Shadow Mapping}

Schatten abbilden

\subsection{Raytracing}

Strahlen verfolgen

\subsection{Shadow Volumes}

Schatten zum Würfel

In naturwissenschaftlichen Veröffentlichungen sollte immer 
ein 'Abstract', eine 'Einleitung' und soetwas wie 'Ergebnisse und
Diskussion' vorhanden sein. Andererseits muss man sich bei den
Abschnitten wie 'Material und Methoden' und der 'Durchführung'
nicht zwingend an die Überschriften halten. 

Wichtig ist nur, dass man eingehens die Mittel, Techniken,
Methoden, vielleicht das mathematische Instrumentarium 
oder den experimentelle Aufbau erwähnt, mit welchem man 
gearbeitet hat und was essentiell zum Verständnis sein könnte.

Wenn Sie den Leser vorbereitet haben, was da kommt, dann können
Sie die große Synthese startet und ihr ganzes Setup mit allen
nötigen Parametern beschreiben, aus denen Sie letztendlich
die Ergebnisse generiert haben. Aus diesen Überlegungen heraus, 
sieht man bereits, dass die Grenzen zwischen den Bereichen 
'Material und Methoden' sowie 'Durchführung' und mitunter 
bis zu den 'Ergebnissen' verschwimmen können.

\subsection*{Ideen zum Lesen aus der Sicht eines Massenkonsumenten}

Da heutzutage enorm viele wissenschaftliche Artikel
eingereicht und veröffentlich werden, und das in zig verschiedenen
Journalen, kann niemand alles lesen und nur wenige haben die Zeit 
sehr viel zu lesen. Und da man als Leser noch andere Dinge im Leben 
vorhat, gibt es ein paar Techniken.
Diese Techniken spiegeln auch ein wenig die Bedeutung der einzelnen 
Abschnitte einer Veröffentlichung wieder.

Das Wichtigste ist natürlich die Überschrift, denn wenn diese außerhalb
der Interessensphäre des Lesers liegt, dann wird der Leser weiter suchen
und eine anderen Artikel heranziehen.

Danach wendet sich der Vielleser dem Abstract zu. Wenn dieses spannend ist,
wird dieser der Veröffentlichung mehr Zeit widmen. In Fächern wie der
Biochemie gibt es sogar Leute, die nur die Abstracts lesen, was durchaus
seine Berechtigung hat. Im diesen Abstracts stehen beispielsweise, 
wie bestimmte Proteine reagieren. Meistens ist das ausreichend für 
die eigene Forschung oder um etwas in Erfahrung zu bringen.
Übrigens sollte man im Internet theoretisch zu allen Veröffentlichung
die Abstracts mit dem Titel und den Autoren finden, die seit Anbeginn 
elektronischer Journals eingereicht wurden. Bei kostenpflichtigen 
Journals ist das Abstract wie der der Klappentext beim Buch und
entscheidet über Kauf oder Nichtkauf.

Die Diskussion könnte man als dritte Anlaufstelle nehmen. Wenn Sie 
die Ergebnisse, die im Abstract vielleicht bahnbrechend wirkten, genau 
beleuchtet wissen wollen, so sollten Sie hier mehr darüber
finden. Als Autor müßte man sich an dieser Stelle entsprechend 
kritisch mit den Ergebnissen auseinandersetzen. Auch ein Ausblick
ist manchmal sehr nett.

Als Viertes findet man noch eine hohe Informationsdichte in
Abbildungen, Diagrammen und Tabellen. Diese müssen so präsentiert werden,
dass man nicht hunderte von Zeilen Text durchforsten muss, damit man Sie versteht.
Dementsprechend braucht es eine Unterschrift bei Abbildungen und Diagrammen 
und einer Überschrift bei Tabellen wie es am Beispiel von Tab.~\ref{tab:falsch}
gezeigt wird. Achsenbeschriftungen, Einheitenangaben
und generell Übersichtlichkeit ist selbstredend zu beachten.
Andernfalls wird der Autor vielleicht als stümperhaft oder mindestens
wenig beflissentlich wahrgenommen.

\begin{table}[t]
\caption{Eine auffällig gefälschte Statistik über die Wissensaufnahme $\xi$
(in Wissenseinheit pro Minute) in Abhänigkeit von der verstrichenen Vorlesungszeit.}
\label{tab:falsch}
\centering
\begin{tabular}{cc}
\rowcolor{dunkelgrau}
Zeit [min] & $\xi$ [WE/min] \\
0 - 15 & 20 \\
\rowcolor{grau}
15 - 30 & 30 \\
30 - 45 & 42 \\
\rowcolor{grau}
45 - 60 & 70 \\
60 - 75 & 50 \\
\rowcolor{grau}
75 - 90 & 84
\end{tabular}
\end{table}

Wenn nun all das für den Leser interessant wirkte und dieser es vielleicht
selber im Detail nachvollziehen möchte, dann wird er sich wahrscheinlich
dem restlichen Text widmen.

Natürlich ist das eben Geschriebene nur eine Ideenskizze und wenn
Sie eine andere Herangehensweise an das Lesen solcher Artikel haben,
so steht dies Ihnen selbstverständlich frei.
