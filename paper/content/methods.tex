\section{Problemstellung und Methoden}

Das Problem ist, im zweidimensionalen Raum realistische Schatten zu berechnen möglichst die, wie echte
Schatten aus den drei Teilen "`Umbra"', "`Penumbra"' und "`Antumbra"'

\subsection{Umbra, Penumbra und Antumbra}

In der Theorie von Schatten wird ein Schatten in 4 Teile zerlegen.
\begin{enumerate}
 \item \emph{Umbra}, der Kernschatten, ist der Teil des Schattens der von keinerlei Licht erreicht wird.
       Wenn es nur eine Lichtquelle gibt, dann hat jeder Schatten einen Umbra.
 \item \emph{Antumbra}, der weiche Schatten, der hinter dem Umbra entsteht. Dieser Bereich entsteht nur,
       wenn die Lichtquelle größer ist, als das Objekt, dass den Schatten wirft. Der Antumbra beginnt an
       dem Punkt, an dem das Objekt die Lichtquelle nicht mehr vollständig verdecken kann und man die
       Ränder der Lichtquelle sehen kann.
 \item \emph{Penumbra}, die zwei weichen Schatten, die an den Seiten des Umbras entstehen. Diesen Bereich
       gibt es immer, wenn der Schatten nicht von einer Punktlichtquelle geworfen wird, in der realen
       Welt als immer. In diesem Bereich beginnt man die Seite der Lichtquelle zu sehen, die vorher im
       Umbra verdeckt wurde.
\end{enumerate}
Diese 3 separaten Bereiche eines Schatten sollten optisch realistisch dargestellt werden können.
Damit gilt also als Anforderung, dass Lichtquellen simuliert werden können, die eine Breite haben
und nicht nur einfach ein Punkt sind.


\subsection{Shadow Mapping}

Schatten abbilden

\subsection{Raytracing}

Strahlen verfolgen

\subsection{Shadow Volumes}

Schatten zum Würfel
