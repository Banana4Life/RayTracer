\section{Abstract}

Im Folgenden werden verschiedene Methoden zur Simulation von realistischen Schatten
vorgestellt und diskutiert. Dabei werden die Verfahren "`Shadow Mapping"',
"`Ray tracing"' und "`Shadow Volumes"' theoretisch und die Implementierung
einer Abwandlung des "`Shadow Volumes"'-Verfahrens praktisch gezeigt.


\section{Einleitung \& Motivation}

Schatten gibt es überall, wo es Licht gibt. Dies gibt es zu bedenken, wenn
Gebäude oder außenbereiche geplant werden.
Desweiteren verleihen Schatten flachen Objekten eine gewisse Tiefe für das menschlische Auge.
Dies wird genutzt um in Filmen digital eingefügte Objekte realistisch aussehen zu lassen.

Aber auch in Spielen werden Schatten auf die verschiedensten Weisen verwendet, dort werden
die Schatten meistens auch in Echtzeit berechnet da sich die Spielwelten in den meisten Spielen
ständig ändern.

Architekten nutzen Licht- und Schattensimulationen um Räume zu planen, bei denen eine passende
Ausleuchtung essentiell ist.

\subsection*{Anwendung in Spielen}

In Spielen trifft man auf viele verschiedene Anwendungen von Licht und Schatten. In den meisten
Spielen werden die Schatten genutzt, um die Spielwelt realistischer und glaubwürdiger darzustellen.
Dabei treten häufig viele verschiedene Lichtquellen in einer Szene auf, was für die Berechnung der
Schatten sehr effiziente Algorithmen benötigt, denn eine Szene (ein Bild) muss für eine Bildrate
von 60 $\frac{\text{Bilder}}{\text{Sekunde}}$ in unter 16 Millisekunden fertig von der Grafikkarte gezeichnet sein.

Zum Zeitpunkt der Veröffentlichung dieser Arbeit ist es üblich, grobe Annäherungen zu verwenden,
wie sie etwa durch das "`Shadow Mapping"'-Verfahren berechnet werden.

Andere Spiele verwenden ähnliche Berechnungen auch für Spielelemente. So wird in dem Spiel "`Monaco"' \cite{monaco2014},
ein 2D Spiel in der Vogelperspektive, der Spieler als gerichtete Lichtquelle betrachtet um verdeckte
Elemente, also Elemente im Schatten des Blicks, auszublenden.

\subsection*{Anwendung in der Architekur}

Architekten simulieren bei neu entworfenen Gebäuden auch den Lichteinfall zu verschiedenen Tageszeiten.
Das ist besonders dann wichtig, wenn Kunden gut ausgeleuchtete Räume wollen oder auch wenn es darum geht
Sitzmöglichkeiten auf Plätzen zu planen. In ersterem will man die Schatten durch geschickte Platzierung
von Fenstern und Lampen minimieren, bei letzterem will man den Schatten zu bestimmten Tageszeiten
möglichst lange halten.

\subsection*{Anwendung in Filmen}

Viele Szenen in modernen Filmen werden heutzutage vollständig am Computer generiert. Damit diese Szenen
real aussehen, müssen auch hier wie bei den Spielen akurate Schatten berechnet werden.
